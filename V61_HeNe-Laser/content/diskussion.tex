\section{Diskussion}
\label{sec:Diskussion}

\subsection{Stabilitätsbedingung}

Die Stabilitätsbedingung konnte für den Resonator bestehend aus zwei konkaven Spiegeln nicht abschließend getestet werden, 
da die optische Bank zu kurz war. Dennoch kann bei einer erreichten Resonatorlänge von $L_{\text{real}} = 217,5 \: cm$ und einer 
theoretisch maximalen Resonatorlänge von $L_{\text{theorie}} = 280 \: cm$ bestätigt werden, dass die maximale Resonatorlännge 
nach unten hin zumindest in der selben Größenordnung ist wie der Theoriewert und nicht deutlich kürzer.
Für den Resonator mit einem planen und einem konkaven Spiegel hingegen wurde eine maximale Resonatorlänge von $L_{\text{real}} = 144,5 \: cm$
erreicht, wohingegen der theoriewert nur eine Länge von $L_{\text{theorie}} = 140 \: cm$ angibt.
Offensichtlich stimmt dieser wert nicht mit der Praxis überein. Der Laser hört anscheinend nicht schlagartig auf zu lasen, 
wenn die Stabilitätsbedingung überschritten ist.


\subsection{TEM-Moden}

Wie in den Graphiken der Auswertung zu sehen entsprechen die gemessenen Intensitätswerte der TEM$_{00}$-Mode gut dem Theoretischen 
Ansatz der Gaußfunktion. 
Auch die Intensitätsverteilung der TEM$_{10}$-Mode entsprechen im wesentlichen dem Ansatz der Theorie.
Allerdings weichen hier die Messwerte stärker von der Ausgleichsfunktion ab als bei der Grundmode. 
Dies könnte daran liegen, dass die gemessenen Intensitäten deutlich geringer waren und dadurch Störquellen wie z.B. das Licht 
der Anzeige der Photodiode einen größeren Einfluss auf die Messungen hatten.
Außerdem fällt auf, dass die Intensitätsverteilung nicht Symmetrisch um das Minimum herum verteilt ist. 
Dies liegt an dem Draht, der die Grundmode stört, dieser war anscheinend nicht komplett waagerecht im resonator plaziert, sondern leicht schräg.


\subsection{Polarisation}

Die Intensität des Lasers hat bei einer Polarisation von $\phi_0 = 89,74^{\circ}$ ein Maximum.
Somit ist der Laser ziemlich genau parallel zum Tisch ploarisiert. 
Auch der restliche Intensitätsverlauf abhängig vom Polarisationswinkel entspricht gut der Theoretischen annahme 
und hat wie erwartet eine 2$\pi$ periodizität.


\subsection{Multimodenbetrieb}

Die gemessenen Frequenzdifferenzen der longitudinalen Moden entsprechen sehr gut den theoriewerten und weichen für jede der 
vier Resonatorlängen um weniger als ein Prozent von der Theorie ab.
Die damit zu vergleichende Doppler-Verschiebung ist mit einem Theoretischen Wert von $\partial f_D = 1308,21$ MHz 
um etwa eine größenordnung größer.


\subsection{Wellenlänge}

Die durch Gitterbeugung bestimmte Wellenlänge des Lasers liegt beim ersten und zweiten Gitter bei 
$\lambda_1 = 705,62 \pm 40,95 \si{\nano\meter}$ und $\lambda_2 = 689,87 \pm 50,41 \si{\nano\meter}$.
Diese werte liegen im bereich der theoretischen Wellenlänge von um die 632,8 nm.
Auffällig sind die beiden großen Fehler. Diese lassen darauf schließen, dass einzelene Messwerte stark vom Mittelwert abweichen. 
Das könnte der Fall sein, wenn die Wand, auf die die Intensitätsmaxima projeziert wurden, nicht senkrecht zur 
Ausbreitungsrichtung des Lasers stand, sondern leicht schräg. Die daraus resultierenden Abweichungen der Wellenlänge führen allerdings 
zu keinem Systematsichen Fehler des Mittelwertes, sondern lediglich zu einem großen Fehler, da sich die Abweichungen im Mittelwert aufheben. 
Bei den beiden Gittern mit den größeren Gitterkonstanten, liegt die bestimmte Wellenlänge um den Faktor drei bis sechs 
neben dem Theoretischen wert. Da diese beiden Gitter mit einem Abstand $l = 30$ cm und nicht wie die beiden anderen mit $l = 50$ cm 
benutzt wurden, liegt nahe, dass im bereich dieser Abstandsmessung ein fehler unterlaufen ist. 
Unabhängig davon sind die Gitter mit kleineren gitterkonstanten in diesem Fall geeigneter zur bestimmung der Wellenlänge dieses Lasers, da sie 
mehr Maxima produzieren und so die Unsicherheit geringer wird.


