\section{Auswertung}
\label{sec:Auswertung}

Zuerst werden die Daten des Detektorscans an eine Gaussfunktion angepasst, um die Halbwertsbreite und die maximale Intensität zu bestimmen. Als Formel für die Gausskurve wurde dabei 
\begin{equation*}
  I(\theta) = \frac{a}{\sigma\sqrt{2\pi}} \exp\left( \frac{-\left( \theta - \theta_0\right)^2}{2 \sigma} \right)
\end{equation*}
genutzt. Die Originaldaten des Detektorscans und die Ausgleichskurve sind in Abbildung \ref{fig:detektor} zu sehen.

\begin{figure}[H]
    \centering
    \includegraphics{build/detektorscan.pdf}
    \caption{Daten des Detektorscans angepasst an eine Gausskurve.}
    \label{fig:detektor}
  \end{figure}
Aus dem Fit ergeben sich die Werte:
\begin{align*}
  a &= \num{3.039(16)e6} \\
  \theta_0 &= \SI{0.0040(1)}{\degree} , \\
  \sigma &= \SI{0.0209(1)}{\degree}\,. \\
  \end{align*}
Aus diesen Werten erhält man für die maximale Intensität $ \frac{a}{\sigma\sqrt{2\pi}}=$ \num{58.030(465)e6} Events als Maximum. 
Die Halbwertsbreite beträgt $2 \sqrt{2 \ln 2} \, \sigma$, also hier \SI{0.0492(3)}{\degree}.

Anschließend wird der diffuse Scan von dem Reflektivitätsscan abgezogen, um die korrigierten Messwerte zu erhalten. Diese sind in Abbildung \ref{fig:messwerte} dargestellt.
  \begin{figure}[H]
    \centering
    \includegraphics{build/messwerte_relativ.pdf}
    \caption{Messwerte des Reflektivitätsscans, des diffusen Scans und die korrigierten Messwerte.}
    \label{fig:messwerte}
  \end{figure}

Als nächstes werden die Werte am Winkel des zweiten Maximums (\SI{0.315}{\degree}) normiert und dann neben der Theoriekurve der Fresnelreflektivität einer ideal glatten Siliziumoberfläche in Abbildung \ref{abb:norm} dargestellt. Dabei wurde als Wert für die Wellenlänge $\lambda = \SI{1.54e-10}{\meter}$ und für den Brechungsindex von Silizium $n = \num{1} - \num{7.6e-6} + \num{1.73i e-9}$ genutzt.
\begin{figure}
    \centering
    \includegraphics{build/silizium.pdf}
    \caption{Normierte Messwerte und Theoriekurve für eine ideal glatte Siliziumoberfläche.}
    \label{abb:norm}
\end{figure}

Im nächsten Schritt der Auswertung wird der Geometriewinkel $\alpha_g$ aus den Daten, die bei der Justierung aufgenommen wurden, ermittelt. Die Messergebnisse aus dem z-Scan und aus dem Rocking-Scan sind in den Abbildungen \ref{fig:z} und \ref{fig:rocking} zu sehen. 
  \begin{figure}[H]
    \centering
    \includegraphics{build/zscan.pdf}
    \caption{Daten des z-Scans.}
    \label{fig:z}
  \end{figure}

  \begin{figure}[H]
    \centering
    \includegraphics{build/rocking.pdf}
    \caption{Daten des Rocking-Scans. (Absorber=100)}
    \label{fig:rocking}
  \end{figure}


Für den Rocking-Scan beträgt der Geometriewinkel also $\alpha_g = \SI{0.4}{\degree}$. Daraus lässt sich zusammen mit dem abgemessenen Durchmesser der Probe ($D = \SI{21.5}{\milli\metre}$) die Strahlhöhe berechnen und es ergibt sich $d_0 = \SI{0.15}{\milli\meter}$ und ein Geometriefaktor von $G = \num{143.33} \cdot \sin \alpha_i$ für alle Winkel $\alpha_i$, die kleiner als der Geometriewinkel $\alpha_g$ sind.
Alternativ lässt sich der Geometriewinkel mit der Gleichung 
\begin{equation*}
  \alpha_g = \arcsin\left( \frac{d}{D} \right)
\end{equation*}
bestimmen, wobei sich die Strahlbreite $d$ aus dem z-Scan in Abbildung \ref{fig:z} zu $d = \SI{0.16}{\milli\meter}$ ablesen lässt.
Damit kann für den Geometriewinkel $\alpha_g = \SI{0.43}{\degree}$ berechnet werden.

Desweiteren wird die Schichtdicke aus den Kiessig-Oszillationen bestimmt. Dafür konnten aus den Messwerten 9 eindeutige Minima abgelesen werden, deren Daten in \autoref{tab:kiessig} aufgelistet sind. Mit Gleichung \autoref{eqn:dicke} lässt sich aus diesen Daten die Schichtdicke bestimmen, die ebenfalls in \autoref{tab:kiessig} aufgelistet ist. Dabei ist zu beachten, dass die gemessenen Werte für die Winkel ins Bogenmaß umgerechnet und halbiert werden müssen.

\begin{table}[H] 
  \caption{Winkel der Minima der Kiessig-Oszillationen.}
    \label{tab:kiessig}
    \centering
  %  \sisetup{round-mode =places, round-integer-to-decimal=true, round-precision=3}
    \begin{tabular}{S[] S[] S[] }
        \toprule
        {$\alpha_i / \si{\degree}$} & {$\alpha_{i+1} - \alpha_{i} / \si{\degree}$} & {$d / \si{\angstrom}$} \\
        \midrule
        0.295 & 0.045   & 980.4  \\
        0.34 & 0.0475   & 928.8  \\
        0.3875 & 0.05   & 882.4  \\
        0.4375 & 0.05   & 882.4  \\
        0.4875 & 0.05   & 882.4  \\
        0.5375 & 0.0525  & 840.2  \\
        0.59 & 0.0525   & 840.2  \\
        0.6425 & 0.0525  & 840.2  \\
        0.695 &    &   \\
        \bottomrule
    \end{tabular}
\end{table}

Im Mittel beträgt die so berechnete Schichtdicke also $\bar{d} = (884.6 \pm 49.4) \text{\AA}$.
