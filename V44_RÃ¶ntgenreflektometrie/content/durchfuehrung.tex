\section{Durchführung}
\label{sec:Durchführung}
Das eigentliche zu benutzende Diffraktometer war defekt, daher wurde ein anderes benutzt.

Vor der eigentlichen Messung muss die Probe Justiert werden. dafür wird der Siliziumwafer per Hand möglichst genau in die Mitte 
zwischen Röntgenröhre und Detektor gelegt. 
Anschließend wird die Pprobe aus dem Strahl heraus gefahren und ein Detektorscan durchgeführt.
Der Messbereich ist -0,5 bis 0,5 mit 0,04er Schrittweite und der Messzeit 1, pro Messpunkt 
Das Maximum dieser Messung wird durch den "ZI"-Knopf bestimmt und anschließend durch die Eingabe als neue Nulllage des Detektors festgelegt.\\

Nun folgt die Justage des Siliziumwafers.
Der Siliziumwafer wird wieder zurück auf seine anfängliche Position in den Strahl gefahren und dann ein Z-Scan durchgeführt.
Der Messbereich liegt von -10,5 bis -8,5 und wird mit einer Schrittweite von 0,04 und einer Messzeit pro Messpunkt von 1, durchlaufen.
Es wird der Mittelpunkt der gemessenen Werte bestimmt und per Doppelklick und bestätigen des "Move Drives" - Buttons zur neuen Z-Position des Wafers gemacht.\\

Daraufhin folgt der dritte Scantyp, ein Rocking-Scan.
Der Winkel $2 \theta$ bleibt auf null eingestellt. Der Messbereich wird auuf -1 bis 1 eingestellt, mit einer Schrittweite von 0,04 und Messzeit pro Messung von 1.
An den entstehenden Graphen wird herangezoomt, um die Messung noch einmal genauer durchzuführen. der Messbereich liegt nun bei -0,3502 bis 0,4998 und die 
Schrittweite bei 0,01. Die Messzeit bleibt gleich. Nun wird durch den "ZI"-Knopf das Maximum der Verteilung bestimmt und durch die Eingabe die neue Nulllage des Detektors 
festgelegt.
Bisher war der Absorber auf "Auto" eingestellt. Dieser wird jetzt auf 100 gestellt und der präzisere Rocking-Scan, als Vergleichswert, widerholt.\\

Der Z-Scan wird ein weiteres mal durchgeführt, nachdem der Absorber wieder auf "Auto" eingestellt wurde.
Der Scan und die anschließende Kalibrierung der Z-Koordinate verlaufen ebenso, wie beim ersten Mal.
Einzig das Intervall reicht von 0,5 unterhalb der aktuellen Z-Koordinate bis 0,5 oberhalb dieser und die Schrittweite beträgt 0,02.\\

Auch der Rocking-Scan wird wiederholt, allerdings wird vorher "Cube" auf 0,15 Grad gestellt, ebenso wie "Detector".
Der Messbereich reicht von 0 bis 0,3 in 0,005er Schritten. Das Maximum wird anschließend durch den "ZI"-Knopf bestimmt 
und im Feld "Enter theoretical Position" 0,15 eingegeben und gespeichert.\\

Der letzte Punkt der Kalibrierung ist wieder ein Z-Scan, der genau wie der zweite Z-Scan abläuft, 
nur dass "Cube" und "Detector" wie im vorherigen Scan auf 0,15 eingestellt sind.
Als neue Z-Koordinate wird das Maximum der gemessenen Kurve gewählt.\\

Für die Messung wird der "Scantyp" "Omega/2Theta" ausgewählt, mit einem Scanbereich von 0° bis 2,5°, einer Schrittweite von 0,005° und einer 
Messzeit von 5 s. Da die auflösung der messung sehr schlecht ist, wird die Messzeit auf 10 s pro Messpunkt erhöht und der Bereich auf 0° bis 1,5° verkleinert.
Für den "Diffusen Scan" wird "Detector" auf 0,1 gestellt und die restliche Messung eben so wie beim ersten mal durchgeführt.



