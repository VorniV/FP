\section{Diskussion}
\label{sec:Diskussion}
Zu Beginn der Auswertung wurde der bei der Justierung aufgenommene Detektorscan an eine Gausskurve angepasst, um die Halbwertsbreite und die maximale Intensität zu bestimmen. Dies hat gut funktioniert und es hat sich eine Gausskurve ergeben, deren Maximum ziemlich genau bei \SI{0}{\degree} liegt. \\
\newline
Dann wurden die Messdaten erstmals korrigiert, indem der diffuse Scan von dem Reflektivitätsscan abgezogen wurde. In den Messdaten lassen sich die Kiessig-Oszillationen gut erkennen, wie in Abbildung \ref{fig:messwerte} zu sehen ist. 
\newline
In Abbildung \ref{abb:norm} wurden die normierten Messdaten sowie die Theoriekurve für eine ideal glatte Siliziumoberfläche dargestellt. Hier ist zu sehen, dass der grundlegende Verlauf beider Kurven zueinander passt, aber auch, dass beide Kurven voneinander abweichen. Die Abweichung ist unter anderem damit zu erklären, dass unsere Probe keine ideal glatte Siliziumoberfläche aufweist, sondern eine gewisse Rauigkeit hat. \\
\newline
Anschließend wurden der Geometriewinkel und Geometriefaktor bestimmt. Aus dem Rocking-Scan konnte dabei ein Geometriewinkel von $\alpha_g = \SI{0.4}{\degree}$ und ein Geometriefaktor von $G = \num{143.33} \cdot \sin \alpha_i$ bestimmt werden. Nachdem die Strahlbreite aus dem z-Scan abgelesen wurde, konnte daraus alternativ ein Geometriewinkel von $\alpha_g = \SI{0.43}{\degree}$ berechnet werden. Die relative Abweichung der beiden bestimmten Geometriewinkel liegt also bei $7.5\%$. 
\newline
Außerdem wurde die Schichtdicke aus den Kiessig-Oszillationen berechnet. Dabei konnte die Schichtdicke zu $d = (884.6 \pm 49.4) \text{\AA}$ bestimmt werden. Dieser Wert liegt in der richtigen Größenordnung und weist einen relativen Fehler von $5.6\%$ auf.