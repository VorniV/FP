\section{Diskussion}
\label{sec:Diskussion}
Der experimentell bestimmte Lande-Faktor beträgt für die rote Cadmium-Linie $g_r = 1,04 \pm 0,05$ und der 
theoretische Wert $g_{r,the} = 1$. Dies entspricht einer Abweichung von 4\%.
Der experimentelle Wert des Lande-Faktors der blaunen $\sigma$-Linie beträgt $g_b = 1,75 \pm 0,06$. Dies entspricht genau dem 
theoretischen Wert von $g_{b,the} = 1,75$. Dieser theoretische Wert setzt sich aus zwei verschiedenen Lande-Faktoren der 
blauen $\sigma$-Linie zusammen. Zum einen $g_{b,the,1} = 1,5$ und zum anderen $g_{b,the,2} =2$. Da die Auflösung der Lummer-Gehrcke-Platte 
nicht gut genug ist um die kleine Aufspaltung dieser zwei Linien sichtbar zu machen, erscheinen sie als eine Linie und der 
Lande-Faktor wird als Mittelwert angenommen.
Die Messergebnisse der beiden Cadmium-Linien scheinen sehr gut zu sein. Die geringen Abweichungen der Lande-Faktoren zu den 
Theoriewerten können vorallem durch die nicht exakte bestimmung der verwendeten Magnetfeldstärken erklärt werden, ebenso wie 
durch die Ungenauigkeit beim bestimmen der Interferenzmaxima, da diese per Augenmaß bestimmt wurden.

Der Lande-Faktor der blauen $\pi$-Kurve konnte nicht bestimmt werden, da wie in Abbildung \ref{fig:blau2} zu sehen, keine Aufspaltung 
der Spektrallinien zu erkennen war. Wie bei den vorherigen Interferenzmustern ist die obere hälfte ohne und die untere Hälfte 
mit Magnetfeld aufgenommen. Die Breite der Spektrallinien bei angeschaltetem Magnetfeld lässt erahnen, 
dass es eine Aufspaltung gabe, die allerdings zu klein war, als dass die von der Lummer-Gehrcke-Platte aufgelöst werden konnte.
Bei der Beobachtung der blauen $\pi$-Linie wurde ein Magnetfeld mit $B = 582,43 \pm 25.57$ verwendet. Das optimale Magnetfeld 
zur Beobachtung dieser Aufspaltung beträgt EINFÜGEN. Somit ist das verwendete Magnetfeld um EINFÜGEN geringer als das optimale, 
was erklären kann, wieso die Wellenlängendifferenz zu klein war um beobachtet werden zu können.

\begin{figure}[H]
    \centering
    \includegraphics[width=\textwidth]{Blau_Pi.png}
    \vspace{-10pt}
    \caption{Interferenzmuster der Cd-Lampe für die blaue $\pi$-Linie, mit und ohne Magnetfeld.}
    \label{fig:blau2}
  \end{figure}

