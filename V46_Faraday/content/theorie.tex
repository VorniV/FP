\section{Theorie}
\label{sec:Theorie}
Ziel des Versuchs ist die Bestimmung der effektiven Masse von Leitungselektronen in einem GaAs-Halbleiter, 
durch den Faraday-Effekt.

\subsection{Bandstrukturen bei Halbleitern}
Einzelne Atome haben diskrete Energieniveaus, die von Elektronen besetzt werden können. 
Festkörper bestehen aus N = $10^{21} - 10^{23}$ Atomen pro cm$^3$, daher bilden sich aufgrund des Pauliverbots Energiebänder.
Diese Energiebänder liegen um die diskreten Energieniveaus der einzelenen Atome herum und sind ebenfalls diskret. 
Aufgrund der großen Anzahl an Atomen und somit der vielen besetzten Zustände pro Band, werden diese oft als kontinuierlich angenommen. 
Die bereiche zwischen den Bändern sind verbotene Zustände und können nicht von Elektronen besetzt werden. 
Bei T = 0 K liegen alle besetzten Zustände eines Halbleiters im Valenzband, im Leitungsband befinden sich keine Elektronen.
Dies hat zurfolge, dass das Material nicht leitet.
Die Bandlücke zwischen Valenzband und Leitungsband ist bei Halbleitern einige wenige eV groß. 
Bei Isolatoren sind die Bandlücken größer, was der einzige Unterschied zwischen Halbleitern und Isolatoren ist.
So kann die Bandlücke eines Halbleiters durch thermische Energie von den Elektronen überwunden werden und der 
Halbleiter wird leitend. Der Anteil der Elektronen, die sich im Leitungsband befinden, wird durch die Boltzmann-Statistik

\begin{equation}
    \frac{N}{N_0} = exp \left(-\frac{\Delta E}{k_B T}\right)
\end{equation} 
beschrieben.
Für die verwendete Probe aus GaAs ergibt dies, mit den genäherten Werten, T = 300 K, $k_B T$ = 26 meV, $\Delta E$ = 1,42 eV, 
einen Elektronenanteil im Leitungsband von $\frac{N}{N_0} \approx 4,12 \cdot 10^{-22}$ \%.
Da diese Anzahl an Elektronen im Leitungsband kaum zum Leiten reicht, werden Halbleiter oft dotiert. 
In diesem Fall ist die GaAs-Probe n-dotiert. Ihr wurden Fremdatome mit einem zusätzlichen Valenzelektron hinzugefügt.
Dieses zusätzliche Elektron ist vom periodischen Potential des Festkörpers weniger stark gebunden als die anderen. 
Der Donatorzustand hat somit eine geringe Bandlücke zum Leitungsband 
und das Elektron benötigt eine geringere Energei, um einen Zustand im Leitungsband besetzen zu können.
Die Bandlücke kann so bis auf wenige Millielektronenvolt reduziert werden, wodurch die Halbleiter auch bei 
Raumtemperatur leitend werden.
Eine weitere Möglichkeit die Bandlücke zu veringern ist die p-Dotierung. 
Dabei werden Fremdatome mit einem Valenzelektron weniger als die des Halbleiters, eingefügt. 
Die Akzeptorzustände haben so einen unbesetzten Zustand, welcher als positiver Ladungsträger, als "Loch" \: beschrieben werden kann. 
Dieses Loch besetzt einen Zustand knapp über dem Valenzband. Wenn es durch äußere Energiezufuhr 
in das Valenzband gelangt, trägt es zur Leitung des Halbleiters bei.

\subsection{Effektive Masse}
Elektronen in einem Kristall bewegen sich quasifrei unter dem Einfluss des periodischen Potentials der Gitteratome.
Da eine Beschreibung von freien Teilchen mit der Energiedispersion $E(\vec{k})$

\begin{equation}
    E(\vec{k}) = \frac{\hbar^2 k^2}{2m},
\end{equation}
mit dem Wellenvektor $\vec{k}$, einfacher ist, wird die effektive Masse $m^*$ eingeführt.
Diese berücksichtigt das periodische Potential und ist definiert als 

\begin{equation}
    m^* = \hbar^2 \left(\frac{\delta^2 E(k)}{\delta k^2}\right)^{-1}.
\end{equation}
So können die Leitungselektronen als freie Teilchen mit der Masse $m^*$ beschrieben werden.
Für drei dimensionale Kristale muss die effektive Masse angepasst werden. 
Jedoch ist die Symmetrie von GaAs hinreicheng hoch, sodass $m^*$ nicht Richtungsabhängig ist und somit 
nicht auf drei Dimensionen angepasst werden muss.

\subsection{Der Faraday-Effekt}
Der Faraday-Effekt beschreibt die Drehung der Polarisationsebene einer linear polarisierten elektromagnetischen Welle, 
in einem optisch neutralem Medium, welches von einem Magnetfeld durchsetzt ist, das in Ausbreitungsrichtung der 
Welle gepolt ist.
Die Drehung der Polarisationsebene in einem optisch aktiven Medium kann mithilfe des Superpositionsprinzips erklärt werden.
So kann eine linear polarisierte Welle $E(z)$ aus zwei eintgegengesetz zirkulierenden Wellen $E_R(z)$, $E_L(z)$ 
beschrieben werden:

\begin{equation}
    E(z) = \frac{1}{2} \left(E_R(z) + E_L(z)\right).
\end{equation}
Mit den links bzw. rechts herum zirkulierenden Wellen

\begin{equation}
    E_R(z) = (E_0 \vec{x_0} - i E_0 \vec{y_0}) e^{i k_R z}
\end{equation}
\begin{equation}
    E_R(z) = (E_0 \vec{x_0} + i E_0 \vec{y_0}) e^{i k_L z},
\end{equation}
dabei gilt $k_R \neq k_L$.
Tritt die Welle nun bei $z = 0$ in ein Medium ein, so ergibt sich für den Drehwinkel $\Theta$ 

\begin{equation}
    \Theta = \frac{d}{2} (k_R-k_L) = \frac{d \omega}{2} \left(\frac{1}{v_{ph,R}} - \frac{1}{v_{ph,L}} \right) = \frac{d \omega}{2 c} (n_R - n_L).
\end{equation}
Mit der Dicke der Probe $d$, den Phasengeschwindigkeiten $v_{ph,i}$ und den Brechungsindizes $n_i$.
Die Rotation entsteht durch Wechselwirkungen der Bandelektronen mit den Atomrümpfen, 
welche pro Volumenelement die Polarisation 
\begin{equation}
    \label{eq:1}
    \vec{P} = \epsilon_0 \chi \vec{E},
\end{equation}
erzeugen.
Dabei ist $\chi$ die elektrische Suszeptibilität, welche für anisotrope Kristalle 
durch den Ansatz
%\begin{equation}
%    \chi = $\begin{array}{rrr} 
%            \chi_{xx} & i\chi_{xy} & 0 \\ 
%            -i\chi_{yx} & \chi_{xx} & 0 \\ 
%            0 & 0 & \chi_{zz} \\ 
%           \end{array}$
%\end{equation}
\begin{equation}
    \chi =
    \begin{pmatrix}
      \chi_{\text{xx}} & i\chi_{\text{xy}} & 0 \\
      -i \chi_{\text{xy}}& \chi_{\text{xx}} & 0 \\
      0& 0 & \chi_{\text{zz}}
      \end{pmatrix} 
\end{equation}
beschrieben wird.
Durch Anwenden der Wellengleichung und Gleichung \eqref{eq:1} kann der Rotationswinkel 
wie folgt ausgedrückt werden:

\begin{equation}
    \Theta \approx \frac{d \omega}{2 c n} \chi_{xy}.
\end{equation}
Um eine Drehung bei optisch inaktiven Medien hervorzurufen, muss ein Magnetfeld 
angelegt werden. 
Diese Rotation kann durch die Bewegungsgleichung 

\begin{equation}
    m \frac{d^2 \vec{r}}{dt^2} + K \vec{r} = - e_0 \vec{E}(r) - e_0 \frac{d \vec{r}}{d t} \times \vec{B},
\end{equation} 
verstanden werden.
Sie beschreibt die Bewegung eines gebundenen Elektrons der Masse $m$ und der Ladung $e_0$ im Magnetfeld $\vec{B}$.
Dabei ist $\vec{r}$ die Auslenkung aus der Gleichgewichtslage, $K$ die Bindungskonstante 
und $\vec{E}(r)$ die Feldstärke der einfallenden elektromagnetischen Welle.
Aus dieser Bewegungsgleichung ergibt sich für den Drehwinkel

\begin{equation}
    \Theta (\lambda) \approx \frac{2 \pi^2 e_0^3 c }{\epsilon_0 m^2 \lambda^2 \omega_0^4} \frac{N B d}{n}.
\end{equation}
Für den Fall des quasifreien Elektrons wird $\omega_0 \to 0$ angenommen 
und die Masse durch die effektive Masse ersetzt. 
Dies ergibt für den normierten Rotationswinkel $\Theta_{\text{frei}} = \frac{\Theta}{d}$

\begin{equation}
    \Theta_{\text{frei}} = \frac{e_0^3}{8 \pi^2 \epsilon_0 c^3} \frac{1}{(m^+)^2} \lambda^2 \frac{N B}{n}.
\end{equation}
Mit $N$ der Anzahl der Ladungsträger pro Volumen, $B$ der manetischen Flussdichte, 
und $\lambda$ der Wellenlänge des einfallenden Lichts.


