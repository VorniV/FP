\section{Theorie}
\label{sec:Theorie}
Ziel des Versuchs ist die Bestimmung der effektiven Masse von Leitungselektronen in einem GaAs-Halbleiter, 
durch den Faraday-Effekt.

\subsection{Bandstrukturen bei Halbleitern}
Einzelne Atome haben diskrete Energieniveaus, die von Elektronen besetzt werden können. 
Festkörper bestehen aus N = $10^21 - 10^23$ Atomen pro cm$^3$, daher bilden sich aufgrund des Pauliverbots Energiebänder.
Diese Energiebänder liegen um die diskreten Energieniveaus der einzelenen Atome herum und sind ebenfalls diskret. 
Aufgrund der großen Anzahl an Atomen und somit der vielen besetzten zustände pro Band, werden diese oft als kontinuierlich angenommen. 
Die bereiche zwischen den Bändern sind verbotene Zustände und können nicht von Elektronen besetzt werden. 
Bei T = 0 K liegen alle besetzten Zustände eines Halbleiters im Valenzband, im Leitungsband befinden sich keine Elektronen.
Dies hat zurfolge, dass das Material nicht leitet.
Die Bandlücke zwischen Valenzband und Leitungsband ist bei Halbleitern einige wenige eV groß. 
Bei Isolatoren sind die Bandlücken größer, was der einzige unterschied zwischen Halbleitern und Isolatoren ist.
So kann die Bandlücke eines Halbleiters durch thermische Energie von den Elektronen überwunden werden und der 
Halbleiter wird leitend. Der Anteil der Elektronen, der sich im Leitungsband befindet, wird durch die Boltzmann-Statistik

\begin{equation}
    \frac{N}{N_0} = exp \left(-\frac{\Delta E}{k_B T}\right)
\end{equation} 
beschrieben.
Für die verwendete Probe aus GaAs ergibt dies, mit den genäherten Werten, T = 300 K, $k_B T$ = 26 meV $\Delta E$ = 1,42 eV, 
einen Elektronenanteil im Leitungsband von $\frac{N}{N_0} \approx 4,12 \cdot 10^{-22}$ \%.
Da diese Anzahl an Elektronen im Leitungsband kaum zum Leiten reicht, werden Halbleiter oft dotiert. 
In diesem Fall ist die GaAs-Probe n-dotiert. Ihr wurden Fremdatome mit einem zusätzlichen Valenzelektron hinzugefügt.
Dieses zusätzliche Elektron ist vom periodischen Potential des Festkörpers weniger stark gebunden als die anderen. 
Der Donatorzustand hat somit eine geringe Bandlücke zum Leitungsband 
und das Elektron benötigt eine geringere Energei, um einen Zustand im Leitungsband besetzen zu können.
Die Bandlücke kann so bis auf wenige Millielektronenvolt reduziert werden, wodurch die Halbleiter auch bei 
Raumtemperatur leitend werden.
Eine weitere Möglichkeit die Bandlücke zu veringern ist die p-Dotierung. 
Dabei werden Fremdatome mit einem Valenzelektron weniger als die des Halbleiters, eingefügt. 
Die Akzeptorzustände haben so einen unbesetzten Zustand, welcher als positiver Ladungsträger, als "Loch" beschrieben werden kann. 
Dieses Loch besetzt einen Zustand knapp über dem Valenzband. Wenn es durch äußere Energiezufuhr 
in das Valenzband gelangt, trägt es zur Leitung des Halbleiters bei.

\subsection{Effektive Masse}



