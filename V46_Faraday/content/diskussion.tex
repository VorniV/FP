\section{Diskussion}
\label{sec:Diskussion}
Als effektive Masse der Leitungselektronen konnte für Probe 1 $m_\text{eff,1} = (\num{0.083(9)})  m_\text{e}$ bestimmt werden und für Probe 2 $m_\text{eff,2} = (\num{0.081(4)}) m_\text{e}$ bestimmt werden. Nach \cite{mass} beträgt der Literaturwert für die effektive Masse der GaAs-Leitungselektronen $m_{\Gamma} = \num{0.063} \, m_\text{e}$. Es ist also zu sehen, dass die ermittelten Werte für die effektiven Massen in der richtigen Größenordnung liegen. Allerdings bleibt eine Abweichung vom Literaturwert von $31.7\%$ für Probe 1 und von $28.6\%$ für Probe 2 zu erkennen. Diese Abweichungen können durch verschiedene Fehlerquellen erklärt werden.\\
\newline
\noindent
So ist zu erkennen, dass die Werte für den vorletzten Filter mit $\lambda=2.51 \si{\micro\meter}$ deutlich abweichen. Es scheint also ein Fehler im Filter vorzuliegen. Dieser fehlerhafte Filter ist eine Erklärung für die Abweichungen. Eine weitere Fehlerquelle kann sein, dass der ermittelte Maximalwert für die Kraftflussdichte des Magnetfeldes von $B = \SI{421}{\milli\tesla}$ nicht der exakte Wert am Ort der Probe ist. Außerdem ist es möglich, dass der gesamte Versuchsaufbau noch besser justiert werden könnte.