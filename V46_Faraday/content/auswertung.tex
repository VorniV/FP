\section{Auswertung}
\label{sec:Auswertung}
In der Auswertung wird zuerst die maximale Kraftflussdichte des B-Feldes ermittelt und anschließend wird die effektive Masse der Leitungselektronen im n-dotierten Galliumarsenid über die gemessenen Winkel mittels Faraday-Rotation bestimmt.

\subsection{Messung der Kraftflussdichte $B(z)$}
Die Messergebnisse für die Kraftflussdichte $B(z)$ in der Nähe des
Luftspaltes bei maximalem Feldstrom mit einer Hallsonde sind im Folgenden aufgelistet:

\begin{table}[H]
    \centering
    \caption{Gemessene Kraftflussdichte B(z) zu verschiedenen Messpositionen}
    \label{tab:bfeld}
    \begin{tabular}{ll}
        \toprule
    z [$\si{\milli\meter}$] & B [$\si{\milli\tesla}$] \\
        \midrule
    100    & 83     \\
    101    & 115    \\
    102    & 163    \\
    103    & 215    \\
    104    & 286    \\
    105    & 335    \\
    106    & 366    \\
    107    & 385    \\
    108    & 398    \\
    109    & 407    \\
    110    & 413    \\
    111    & 417    \\
    112    & 420    \\
    113    & 421    \\
    114    & 421    \\
    115    & 419    \\
    116    & 416    \\
    117    & 412    \\
    118    & 405    \\
    119    & 395    \\
    120    & 381    \\
    121    & 362    \\
    122    & 335    \\
    123    & 297    \\
    124    & 251    \\
    125    & 198   \\
        \bottomrule    
    \end{tabular}
    \end{table}

Diese Messwerte sind in \autoref{fig:bfeld} graphisch dargestellt.

\begin{figure}[H]
    \centering
    \includegraphics[width=\textwidth]{build/bfeld.pdf}
    \caption{Kraftflussdichte B(z) abhängig von der Messposition z.}
    \label{fig:bfeld}
\end{figure}

Die maximale Kraftflussdichte des Magnetfeldes beträgt also
\begin{equation*}
    B = \SI{421}{\milli\tesla}.
\end{equation*}

\subsection{Bestimmung der effektiven Masse mittels Faraday Rotation}
In diesem Abschnitt wird die effektive Masse der Leitungselektronen von Galliumarsenid bestimmt. In \autoref{tab:proben} sind die Werte zu den 3 genutzten Proben aufgelistet.
\begin{table} \caption{Die Werte für die Dicken und die Donatorenkonzentrationen für die reine GaAs-Probe und die beiden n-dotierten GaAs-Proben.}
    \label{tab:proben}
    \centering
    \begin{tabular}{S[] S[] S[] S[]}
        \toprule
        {} & {Reine Probe} & {Probe 1} & {Probe 2} \\
        \midrule
        \text{Dicke} $d / \si{\milli\meter}$ & 5.11 & 1.36 & 1.296 \\
        \text{Donatorendichte} $N / \si[per-mode=reciprocal]{\per\cubic\centi\meter}$ &  & 1.2e18 & 2.8e18 \\
        \bottomrule
    \end{tabular}
\end{table}

In \autoref{tab:winkel} sind die Messergebnisse für die Winkel für die 3 Proben zu den 9 Wellenlängen aufgelistet.
\begin{table}[H]
    \centering
    \caption{Messwerte für die 3 Proben bei 9 verschiedenen Filtern.}
    \label{tab:winkel}
    \begin{tabular}{lllllll}
        \toprule
        Wellenlänge Filter [$\si{\micro\meter}$] & \multicolumn{2}{l}{Winkel Probe 1 [$\si{\degree}$,']} & \multicolumn{2}{l}{Winkel Probe 2 [$\si{\degree}$,']}  & \multicolumn{2}{l}{Winkel reine Probe [$\si{\degree}$,']} \\
        \midrule
    1.06  & 84,20 & 74,26 & 85,25 & 74,0  & 90,25 & 67,40 \\
    1.29  & 80,50 & 74,8  & 82,38 & 73,30 & 86,20 & 70,5  \\
    1.45  & 83,5  & 75,53 & 83,40 & 74,46 & 85,50 & 72,50 \\
    1.72  & 80,0  & 74,6  & 82,0  & 71,20 & 82,15 & 73,0  \\
    1.96  & 75,0  & 68,50 & 77,40 & 66,39 & 75,40 & 69,5  \\
    2.156 & 73,20 & 67,28 & 76,0  & 64,0  & 73,25 & 68,0  \\
    2.34  & 10,35 & 4,0   & 53,0  & 39,20 & 48,32 & 43,50 \\
    2.51  & 25,45 & 26,30 & 28,35 & 22,40 & 24,4  & 28,25 \\
    2.65  & 67,0  & 58,47 & 71,31 & 57,25 & 63,18 & 58,46 \\
    \bottomrule
    \end{tabular}
\end{table}

Wie in \autoref{tab:winkel} zu sehen ist, wurden zu jedem Filter und zu jeder Probe 2 Winkel aufgenommen. Um aus diesen 2 Winkeln einen auf die Länge skalierten Winkel zu erhalten, wird die Formel 
\begin{equation}
    \theta_{\text{KR}} = \frac{\theta_1 - \theta_2}{2d}
\end{equation}
genutzt. So ergeben sich die folgenden skalierten Winkel $\theta_{\text{KR}}$:

\begin{table}[H]
     \caption{Skalierte Rotationswinkel der drei Proben für die verschiedenen Wellenlängen.}
    \label{tab:skaliert}
    \centering
    \begin{tabular}{cccc}
        \toprule
        {$\lambda / \si{\micro\meter}$} & {$\theta_{\text{KR}} \text{[Reine Probe]}/ \si[per-mode=fraction]{\radian\per\meter}$} & {$\theta_{\text{KR}} \text{[Probe 1]}/ \si[per-mode=fraction]{\radian\per\meter}$} & {$\theta_{\text{KR}} \text{[Probe 2]}/ \si[per-mode=fraction]{\radian\per\meter}$} \\
        \midrule
     1.06 &  38.85 &  63.52   &   76.87 \\
     1.29 &  27.75 &  42.99   &   61.50 \\
     1.45 &  22.20 &  46.20   &   59.93 \\
     1.72 &  15.79 &  37.85   &   71.82 \\
     1.96 &  11.24  & 39.56  &   74.18 \\
     2.156&  9.25  &  37.64   &   80.80 \\
     2.34 &  8.03  &  42.24   &   92.02 \\
     2.51 &  -7.43 &  -4.81   &   39.84 \\
     2.65 &  7.74 &  52.72   &   94.94 \\
        
        \bottomrule
    \end{tabular}
\end{table}

In \autoref{fig:winkel} sind die skalierten Winkel gegen die quadratische Wellenlänge aufgetragen.
\begin{figure}[H]
    \centering
    \includegraphics[width=\textwidth]{build/winkel.pdf}
    \caption{Skalierte Winkel zur reinen und zu beiden n-dotierten Proben aufgetragen gegen die quadratische Wellenlänge.}
    \label{fig:winkel}
\end{figure}


Im nächsten Schritt wird jeweils die Differenz des Rotationswinkels der reinen Probe und der beiden dotierten Proben gebildet, um die effektive Masse bestimmen zu können. So ergeben sich die folgenden Werte:
\begin{table}[H] 
    \caption{Die Wellenlänge ist gegen die Differenz zwischen der reinen Probe und den beiden n-dotierten Proben aufgelistet.}
    \label{tab:frei}
    \centering
    \begin{tabular}{S[] S[] S[]}
        \toprule
        {$\lambda / \si{\micro\meter}$} & {$\theta_\text{frei,1}/ \si[per-mode=fraction]{\radian\per\meter}$} & {$\theta_\text{frei,2}/ \si[per-mode=fraction]{\radian\per\meter}$} \\
        \midrule
1.06    & 24.67  & 38.02 \\
1.29    & 15.24  & 33.74 \\
1.45    & 23.99 & 37.72 \\
1.72    & 22.06 & 56.02 \\
1.96    & 28.33  & 62.93 \\
2.156   & 28.39  & 71.55 \\
2.34    & 34.22  & 83.99 \\
2.51    & 2.62  & 47.26 \\
2.65    & 44.98  & 87.20 \\
        \bottomrule
    \end{tabular}
\end{table}
In \autoref{fig:probe1} und in \autoref{fig:probe2} sind diese Werte abhängig von der quadratischen Wellenlänge zusammen mit einer entsprechenden Ausgleichsgeraden dargestellt. 

\begin{figure}[H]
    \centering
    \includegraphics[width=\textwidth]{build/Probe1.pdf}
    \caption{Differenz der Rotationswinkel der reinen Probe und Probe 1 aufgetragen gegen die quadratische Wellenlänge mit einer Ausgleichsgeraden.}
    \label{fig:probe1}
\end{figure}

\begin{figure}[H]
    \centering
    \includegraphics[width=\textwidth]{build/Probe2.pdf}
    \caption{Differenz der Rotationswinkel der reinen Probe und Probe 2 aufgetragen gegen die quadratische Wellenlänge mit einer Ausgleichsgeraden.}
    \label{fig:probe2}
\end{figure}

Als Steigungen der beiden Ausgleichsgeraden ergeben sich so:
\begin{align*}
    a_1 &= (5.4 \pm 1.1)*10^{12} \frac{\mathrm{rad}}{\mathrm{m^3}} \\
    a_2 &= (13.3 \pm 1.4)*10^{12} \frac{\mathrm{rad}}{\mathrm{m^3}}
\end{align*}

Aus diesen Steigungen kann dann mit \autoref{eq:mass} die effektive Masse bestimmt werden. So folgt für die effektiven Massen
\begin{align*}
    m_\text{eff,1} &= (\num{0.083(9)}) \, m_\text{e} \\
    m_\text{eff,2} &= (\num{0.081(4)}) \, m_\text{e},
\end{align*}
wobei $m_\text{e}$ der Elektronenmasse entspricht.